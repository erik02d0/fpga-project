\documentclass[a4paper,11pt]{article}

\begin{document}

\section{What}

We're going to talk about Chisel, which stands Constructing Hardware In a Scala Embedded Language. Chisel was developed at UC Berkeley around 2012 and is, like the acronym says, embedded in the Scala programming language. Scala is a strongly typed language with support for object-oriented programming and functional programming, and compiles to Java bytecode, meaning it runs on the Java Virtual Machine. Chisel has received funding from a number of major players, including Microsoft, Intel, Nvidia, Oracle, and others, as well as DARPA (ARPANET, VLSI, ...).

\section{Why (1/2)}

So what's the motivation for developing something like Chisel? Well, VHDL and Verilog were designed some $30+$ years ago for the purpose of simulation, not synthesis. Thus, VHDL and Verilog lack the powerful abstraction facilities that are common in modern software languages (we'll get to this in a bit).

Now, to work around these limitations, you can either use some other language (like Perl) as a macro processing language for your HDL of choice. Or, start out from a domain-specific language and generate hardware blocks from there.

\section{Why (2/2)}

But, combining an HDL with another language is awkward and tricky, and only serves to combine the poor abstraction facilities of the underlying HDL with a completely different high-level programming model that doesn't understand hardware types and semantics. Abstraction is an important aspect of Chisel, because that is what makes it possible for users to conveniently create reusable objects and functions, to define their own data types, and to better capture particular design patterns by writing their own domain-specific languages on top of Chisel - something the developers hope will lead to an explosion in languages and compilers.

As for domain-specific languages as a fix, while they provide greater designer productivity, that's only if the task matches the pattern encoded in the application programming model.

A key motivation for embedding Chisel in Scala is to support highly parameterized circuit generators, a weakness of traditional HDLs. Chisel also supports recursive creation of hardware subsystems. Verilog is not able to describe this type of recursion.

\section{Design costs dominate}

Further motivation for creating Chisel is that the currently dominating design paradigm is slow and expensive. A lot of people are needed for the design, testing and verification processes, each of which is a slow process, using costly proprietary tools. All this adds up to costs in the tens of millions of dollars.

Fast simulation is crucial to reduce hardware development time. Custom logic simulation engines can provide fast cycle-accurate simulation, but are generally too expensive to be used by individual designers.

\section{What it does}

Chisel is intended to be a simple platform that provides modern programming language features for accurately specifying low-level hardware blocks, but which can be readily extended to capture many useful high-level hardware design patterns.

By using a flexible platform, each module in a project can employ whichever design pattern best fits that design, and designers can freely combine multiple modules regardless of their programming model.

Embedding Chisel in Scala raises the level of hardware design abstraction by providing concepts including object orientation, functional programming, parameterized types, and type inference.

Chisel can generate a high-speed C++ based cycle-accurate software simulator, or low-level Verilog designed to map to either FPGAs or to a standard ASIC flow for synthesis.

\section{Some details}

The basic Chisel datatypes are used to specify the type of values held in state elements like registers or flowing on wires. (The Bits, Fix, UFix and other datatypes are distinct from Scala's builtin types.)

Circuits are represented as graphs of nodes. Each node is a hardware operator that has zero or more inputs and that drives one output. Chisel "components" are similar to Verilog modules.

As mentioned earlier, Chisel also supports recursive creation of hardware subsystems. Verilog is not able to describe this type of recursion.

The Chisel compiler produces a C++ class for each Chisel design, with a C++ interface including clock-low and clock-high methods.

The speed of generated simulators is due to using the host machine's architecture; if the RTL bit vector fits into the host's native word size, we lose some overhead. Also, as much branching as possible is removed so as to utilize ILP as much as possible and reduce the number of stalls in the pipeline.

\section{Promises}

The creators of Chisel have highlighted designs where the necessary Chisel code was a third of the "equivalent" Verilog code: a $3$-stage, $32$-bit RISC processor was written in $3020$ lines of Verilog but only $1046$ lines of Chisel. Also, the Chisel-generated circuitry produced less logic area and total area, and simulation times were down by a factor of $8$.

However, a different team did not see much of these improvements in writing a financial asset volatility accelerator, getting only $30\%$ less code (compared to the VHDL version) and about the same amount of hardware. Simulation with embedded IP cores has also proved difficult.

\end{document}



















