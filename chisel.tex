\documentclass{beamer}

\usepackage[latin1]{inputenc}
\usepackage{booktabs}

\usetheme{Warsaw}

\title{Chisel HDL}
\author{Ho-Sung Lee, Erik Vesterlund, Andreas Wennerberg}

\begin{document}

\begin{frame}
\titlepage
\end{frame}

\begin{frame}
\begin{itemize}

\item Developed by UC Berkeley researchers ca 2012

\item Chisel is embedded in Scala

\end{itemize}
\end{frame}

\begin{frame}

Scala is a GP strongly typed programming language, including support for OOP and functional programming. Scala compiles to Java bytecode.

\end{frame}

\begin{frame}
\begin{itemize}

\item Embedding Chisel in Scala raises the level of hardware design abstraction by providing concepts including object orientation, functional programming, parameterized types, and type inference.

\item Chisel can generate a high-speed C++-based cycle-accurate software simulator, or low-level Verilog designed to map to either FPGAs or to a standard ASIC flow for synthesis.

\end{itemize}
\end{frame}

\begin{frame}
\begin{itemize}

\item VHDL and Verilog were designed to be used for simulation, not synthesis

\item VHDL and Verilog lack the powerful abstraction facilities that are common in modern software languages.

\item To work around these limitations one can use another language (e.g. Perl) as a macro processing language for an underlying HDL; alternatively, begin from a domain-specific application programming langugage from which a hardware block is generated.

\end{itemize}
\end{frame}

\begin{frame}
\begin{itemize}

\item The approach of combining VHDL or Verilog with another language is cumbersome, combining the poor abstraction facilities of the underlying HDL with a completely different high-level programming model that does not understand hardware types and semantics.

\item The approach of using a domain-specific langugage provides great designer productivity when the task in hand matches the pattern encoded in the application programming model, but are a poor match for tasks outside their domain.

\end{itemize}
\end{frame}

\begin{frame}
\begin{itemize}

\item Chisel is intended to be a simple platform that provides modern programming language features for accurately specifying low-level hardware blocks, but which can be readily extended to capture many useful high-level hardware design patterns.

\item By using a flexible platform, each module in a project can employ whichever design pattern best fits that design, and designers can freely combine multiple modules regardless of their programming model.

\item Chisel can generate fast cycle-accurate C++ simulators for a design, or generate low-level Verilog suitable for either FPGA emulation or ASIC synthesis with standard tools.

\end{itemize}
\end{frame}

\begin{frame}
\begin{itemize}

\item The basic Chisel datatypes are used to specify the type of values held in state elements or flowing on wires. (The \texttt{Bits}, \texttt{Fix}, \texttt{UFix} and other datatypes are distinct from Scala's builtin types.)

\item Circuits are represented as graphs of nodes. Each node is a hardware operator that has zero or more inputs and that drives one output.

\item Chisel "components" are similar to Verilog modules

\end{itemize}
\end{frame}

\begin{frame}
\begin{itemize}

\item Abstraction is an important aspect of Chisel.

\item Abstraction allows users to coveniently create reusable objects and functions, to define their own data types, and to better capture particular design patterns by writing their own domain-specific languages on top of Chisel.

\end{itemize}
\end{frame}

\begin{frame}
\begin{itemize}

\item A key motivation for embedding Chisel in Scala is to support highly parameterized circuit generators, a weakness of traditional HDLs.

\item Chisel also supports recursive creation of hardware subsystems. Verilog is not able to describe this type of recursion.

\end{itemize}
\end{frame}

\begin{frame}
\begin{itemize}

\item Fast simulation is crucial to reduce hardware development time. Custom logic simulation engines can provide fast cycle-accurate simulation, but are generally too expensive to be used by individual designers.

\item The Chisel compiler produces a C++ class for each Chisel design, with a C++ interface including clock-low and clock-high methods.

\item Simulator code generation is based on templated C++ multiword bit-vector runtime library that executes all the basic Chisel operators.

\item Overhead is removed for the case where the RTL bit vector fits into the host machine's native word size.

\item As much branching as possible is removed so as to best utilize ILP and minimize the number of stalls.

\end{itemize}
\end{frame}

\begin{frame}
\begin{itemize}

\item A 3-stage 32-bit RISC processor was written in 3020 lines of Verilog, and only 1046 lines of Chisel.

\item Chisel produced less logic and total area than Verilog.

\item Simulation was sped up by a factor of 8.

\end{itemize}
\end{frame}

%%%%%%%%%%%%%%% SOURCES
\begin{thebibliography}{9}

\bibitem{tagForThisEntry}
	J. Bachrach et al,
	\emph{Chisel: Constructing Hardware in a Scala Embedded Language},
	2012

\end{thebibliography}

\end{document}



















