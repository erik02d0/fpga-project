\documentclass{beamer}

\usepackage[latin1]{inputenc}
\usepackage{booktabs}

\usetheme{Warsaw}

\title{Chisel HDL}
\author{Ho-Sung Lee, Erik Vesterlund, Andreas Wennerberg}

\begin{document}

\begin{frame}
\titlepage
\end{frame}

\begin{frame}
\frametitle{What?}
\begin{itemize}

\item Developed by UC Berkeley researchers ca 2012

\item Chisel is embedded in the Scala programming language

\item Scala is strongly typed with support for OOP and functional programming. Runs on JVM

\end{itemize}
\end{frame}

\begin{frame}
\frametitle{Why? (1/2)}
\begin{itemize}

\item VHDL and Verilog were designed for simulation, not synthesis

\item VHDL and Verilog lack the powerful abstraction facilities common in modern software languages

\item Workaround: use SW language for macro processing together with HDL, or generate HW using DSL

\end{itemize}
\end{frame}

\begin{frame}
\frametitle{Why? (2/2)}
\begin{itemize}

\item HDL+SW language combines poor abstraction of HDL with completely different programming model

\item DSL increases productivity but only if task and DSL match well

\item HDLs weak on circuit generators

\item Verilog can't support recursive creation

\end{itemize}
\end{frame}

\begin{frame}
\frametitle{Design costs dominate}
\begin{itemize}

\item Synthesis is slow

\item Design, testing and verification are expensive

\item Custom logic sim engines are fast but expensive

\end{itemize}
\end{frame}

\begin{frame}
\frametitle{What it does}
\begin{itemize}

\item Chisel a simple platform with modern language features, extensible

\item Flexible platform, choose design pattern that best fits design

\item Embedding in Scala raises level of HW design abstraction

\item Can generate high-speed, cycle-accurate SW simulator, or Verilog for FPGA/ASIC

\end{itemize}
\end{frame}

\begin{frame}
\frametitle{Some details}
\begin{itemize}
\item Basic Chisel data types used to specify type of values. Types are distinct from Scala's built-ins

\item Circuits represented as graphs of nodes.

\item "Components" are similar to Verilog modules

\item Recursive creation of subsystems

\item One C++ class for each design

\item Sim speed achieved by taking advantage of host architecture, and minimizing branches

\end{itemize}
\end{frame}

\begin{frame}
\frametitle{Promises}
\begin{itemize}
\item Up to $3\times$ greater code density.

\item "Smarter" output (less logic area and total area)

\item Up to $8\times$ faster simulations

\item In practice: Not always that great
\end{itemize}
\end{frame}

%%%%%%%%%%%%%% UNSORTED %%%%%%%%%
\begin{frame}
\begin{itemize}

\item 

%\item Abstraction is an important aspect of Chisel.

%\item Abstraction allows users to coveniently create reusable objects and functions, to define their own data types, and to better capture particular design patterns by writing their own domain-specific languages on top of Chisel.

%\item Simulator code generation is based on templated C++ multiword bit-vector runtime library that executes all the basic Chisel operators.

\end{itemize}
\end{frame}

%%%%%%%%%%%%%%% SOURCES
\begin{thebibliography}{9}

\bibitem{tagForThisEntry}
	J. Bachrach et al,
	\emph{Chisel: Constructing Hardware in a Scala Embedded Language},
	2012

\bibitem{tag2}
	J. Bachrach,
	\emph{Chisel Quick Tutorial},
	1st RISC-V Workshop Proceedings,
	2015

\end{thebibliography}

\end{document}



















